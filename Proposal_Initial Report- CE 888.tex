\documentclass[a4paper]{article}

%% Language and font encodings
\usepackage[english]{babel}
\usepackage[utf8]{inputenc}
\usepackage[T1]{fontenc}

%% Sets page size and margins
\usepackage[a4paper,top=3cm,bottom=2cm,left=3cm,right=3cm,marginparwidth=1.75cm]{geometry}

%% Useful packages
\usepackage{amsmath}
\usepackage{graphicx}
\usepackage[colorinlistoftodos]{todonotes}
\usepackage[colorlinks=true, allcolors=blue]{hyperref}

\title{TWEETEVAL for Emotion Recognition, Emoji Prediction and Irony  Detection}
\author{Muhammad Ilyas\\Registration Number: 2010392 \\School of Computer Science and Electronic Engineering
}
\setlength{\marginparwidth}{2cm}
\begin{document}
\maketitle
\section{Abstract}
\quad Twitter is a major social media platform used by over 330 million people across the globe. Tweets are shared from twitter handles of subscribers on topics of their respective interests and accessed by everyone without bars of geography, ethnicity or linguistics. “Hashtags trends” further propels ideas on a specific topic, though for a brief duration of time. At the same time, Natural Language Processing (NLP) have no dedicated library tools to predict / evaluate contents of the tweets to predict/ map general pulse. This may, potentially, lead to social media abuse against an individual, group or activity based on ethnicity, geography, linguistics or other discriminatory fault line. In this regard, to carry forward the research done at Cardiff university on ‘TWEETEVAL’, this paper will evaluate contents of the tweets for Emojis, Emotions and Irony Detection. A code will be developed to re-train a model, Roberta-base, and carry forward the work done at Cardiff University to enhance the results and achieve better evaluation.  
\section{Introduction}
\quad Social media has become ubiquitous and important for social networking and content sharing. At the same time, it has construed in a manner which promotes public opinion to influence real world outcomes[1] .  It has expanded in space where people can socialize and has become integral part rather than a virtual space where people can only share content[2]. Social media uses various platforms, i.e., Facebook, twitter, Instagram and Pinterest etc to help people share their ideas/ content. The growing use of social media is increasing in capacity to influence outcome of real-world issues. Thus, it is imperative to develop evaluation tools/ mechanism to predict social media trends and their subsequent outcome on daily life issues.

In this backdrop, the paper shall focus on Twitter, a major social media network tool, used by around 330 million users across the globe. 
\section{Background / Literature Review}
\quad Social media is enriched with multifarious tools that are used by subscribers in their routine conversations/ contents. The existing Natural Language Processing (NLP) does not possess requisite capacity to model the existing tools and thus map trends / inclinations from it. Seven twitter tools in question are emotion recognition[3], emoji prediction[4] , hate speech detection[5] , irony detection[6] , offensive language identification[7] , sentiment analysis[8] , and stance detection[9]. 

Basic emotions are four types, i.e. anger, fear, joy and sadness, though their intensity and valence can vary. Emotion recognition was further classified into Emotion intensity regression (EI, reg), Emotion Intensity Ordinal Classification (EIoc), Valence (Sentiment) Regression (V-reg), Valence Ordinal Classification (V-oc), Emotion Classification (E-c). 

Historically, Emojis were first used as “picture characters” in Japan in mobile phone text messages in 1990s. Off late, with the advent and evolution of smartphones, its use became widespread as smartphones support input on a varying range of Emojis.  Presently emojis are widely used by all social media services and messaging platforms. 

In reference paper [5], the authors have discussed Hate Speech as an act, which normally disparages a person and have contents which are having aggressive or harassing attitude towards the target basing upon race, color, ethnicity, gender, sexual orientation, nationality and religion etc. The authors have further described that the algorithm (developed) is having capability to differentiate between incitement against an individual or a group[10].

Irony, by definition, is a literary technique or expression that signifies the actual meaning of a phrase or emotions but on the contrary conveys a radically different or at times opposite meaning of the enunciated expression.  

In reference paper [7], initially the content is evaluated to determine whether it is offensive or non-offensive. In second task, the focus remains to determine the type of the offensive content. In the last task, it decides upon the target of the offensive posts.  

Sentiments are views or opinions held or expressed. In reference paper [8], sentiment analyses involve detecting whether a piece of text expresses a POSITIVE, a NEGATIVE, or a NEUTRAL sentiment; the sentiment can be general or about a specific topic, e.g., a person, a product, or an event.

Stance is the stand or posture adopted about an event, idea or task. In twitter, stance detection is the task of automatically determining from text whether the author of the text is in favor of, against, or neutral towards a proposition or target.

As per the dictates of the problem stated in the academic paper, the paper will focus on three tools to develop model for mapping the use of the same in twitter samples to predict their pattern and accordingly their influence. 
\section{Methods}
\quad The research will be focused on training a model to evaluate the contents of tweets and identify presence of emojis, emotions and irony.
The Roberta-base pre-trained model will be applied on all three data sets to ascertain presence of content in Tweets on Emotion Recognition, Emoji Prediction and irony detection. Further research will be carried out to improve the evaluation matrix and achieve better results than the leader board.  
\section{Results and Discussion}
\quad The Roberta-base model will be re-trained to categorize tweets from the Cardiff university data sets, i.e. TWEETEVAL data sets of emoji, emotions and irony. The accuracy, precision, recall and F score of own model will be compared with the same metrics of Cardiff University model, published on leader boards. 
\section{Conclusion}
\quad At the time of writing of this report, findings of the paper are in preliminary stages. As I progress in formulating/ finalizing the model/ code, the results are expected to improve and can be presented for evaluation. 
\cleardoublepage
\section{Plan}
\begin{figure}[h!]
\centering
\includegraphics[width=1.0\textwidth]{gantt.jpg}
\caption{\label{fig:frog}{\textbf{\textit{Timeline of work in phases}}\textit{}} The X-axis show the timeline by which an activity concludes and the next activity begins, in a phased manner. The Y-axis enumerates the activity/ phase of the project.}
\end{figure}
\begin{thebibliography}{00}
\bibitem{b1} Sitaram Asur and Bernardo A. Huberman, "Predicting the Future with Social Media" published in  2010 IEEE/WIC/ACM International Conference on Web Intelligence and Intelligent Agent Technology. Available at: https://ieeexplore.ieee.org/stamp/stamp.jsp?tp=&arnumber=5616710.
\bibitem{b2} Miller and et al. "How the World Changed Social Media", Available at https://library.oapen.org/handle/20.500.12657/32834.
\bibitem{b3} Saif M. Mohammad, Felipe Bravo-Marquez, Mohammad Salameh and Svetlana Kiritchenko. "Affect in Tweets", Available at https://www.aclweb.org/anthology/S18-1001.pdf (SemEval-2018 Task 1)
\bibitem{b4} Francesco Barbier and et al. Multilingual Emoji Prediction, Available at https://www.aclweb.org/anthology/S18-1003.pdf.
\bibitem{b5} Valerio Basile and et al. Multilingual Detection of Hate Speech Against Immigrants and Women in Twitter, Available at https://iris.unito.it/retrieve/handle/2318/1723924/512658/S19-2007.pdf.
\bibitem{b6} Cynthia Van Hee, Els Lefever and Veronique Hoste. "Irony Detection in English Tweets", Available at https://www.aclweb.org/anthology/S18-1005.pdf (Irony Detection in English Tweets)
\bibitem{7} Marcos Zampieri and et al. Identifying and Categorizing Offensive Language in Social Media (OffensEval), Available at https://www.aclweb.org/anthology/S19-2010.pdf.
\bibitem{b8} Sara Rosenthal, Noura Farra and Preslav Nakov. "Sentiment Analysis in Twitter", Available at https://www.aclweb.org/anthology/S17-2088.pdf (Sentiment Analysis in Twitter)
\bibitem{b9} Svetlana Kiritchenko and et al. "Detecting Stance in Tweets", Available at https://www.aclweb.org/anthology/S16-1003.pdf (Detecting Stance in Tweets) 
\bibitem{b10} Umashanthi Pavalanathan and Jacob Eisenstein. "Emoticons vs. Emojis on Twitter: A Causal Inference Approach", Available at https://arxiv.org/pdf/1510.08480.pdf
\end{thebibliography}
\end{document}
